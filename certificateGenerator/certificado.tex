\documentclass[10pt,a4paper]{article}

\usepackage[landscape,left=3cm,right=3cm,top=3cm,bottom=3cm]{geometry}
\usepackage[brazilian]{babel}
\usepackage[utf8]{inputenc}
\usepackage[T1]{fontenc}

\usepackage{setspace}
\usepackage{datatool}
\usepackage{graphicx}
\usepackage{eso-pic}

\begin{document}

%----------dicionar database----------
\DTLloaddb{database}{database.csv}
\DTLforeach{database}{
  \cpf=cpf, \nome=nome, \telefone=telefone, \emailp=emailp, \palestrante=palestrante, 
  \evento=evento, \datae=datae, \hora=hora, \codigoq=codigoq
}{

  %----------Colocar borda/fundo----------
  \AddToShipoutPicture{
    \centering
    \includegraphics[width=297mm]{moldura.png}
  }

  %----------Adicionar a logo----------
  \begin{figure}
    \centering
    \includegraphics[width=20cm]{logo.png}
  \end{figure}

  %----------Adicionar o titulo----------
  \begin{center}
    \centering
    \bfseries
    \resizebox{!}{1cm}{Certificado}
  \end{center}


  %----------Corpo do Certificado----------
  \vspace{0.5cm}
  \onehalfspacing
  \noindent{
    \bfseries Certificamos que {\nome}, inscrito no CPF \cpf, participou da palestra \textit{\evento}
    ministrado por \palestrante, com duração de \hora.\codigoq
    
    \begin{flushright}
      \noindent \datae
    \end{flushright}
    
  }
  \noindent

  \begin{center}
  \centering
    Este certificado foi gerado automaticamente pela certifico-te.\\
    Sua autenticidade pode ser verificada em nosso site www.certifico-te.com.br\\
    Use o seguinte código: \codigoq
  \end{center}
  
  \thispagestyle{empty}
  \pagebreak

}

\pagebreak
\end{document}